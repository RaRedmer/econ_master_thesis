\section{Discussion}


When discussing these findings, one has to consider the limits of arbitrage. 
In our application, they arise mostly from the microstructure of the crypto market. 
Due to the high trading frequency, the trading strategy is relatively often confronted with liquidity issues.
Therefore, in order to facilitate a trade, 
we followed \cite{gatev2006pairsTrading}, \cite{bowen2016equityMarket} and \cite{liu2015highFrequency}, 
and demand that to only trade (i) when volume is
present for a coin and (ii) with a one period gap after signal generation. 
However, when scaling the invested amount in a position, these restrictions are likely
not enough for simulating actual trading on the crypto exchange. 
The greater the order, the longer it takes for the market to clear it. 
This effect gets even more amplified in markets with low activity and market capitalization.
Further, due to the unregulated nature of crypto markets, 
it is possible to influence markets with large enough orders, 
especially in markets with low capitalization. 
This effect is difficult to model, since model also needs to take its own actions into account.
A possible solution would be to use an agent based model with profits as reward which 
could be realized by Deep Reinforcement Learning. 
Before deploying this method, the microstructure of the market would be rebuild similar to a game.
This market would serve as the environment with which the agent of the Deep Reinforcement Learning algortihm
interacts with.
Another major limit to arbitrage is capacity. An intraday
strategy for cryptocurrencies may offer high returns. By contrast, costs for operating
such a strategy would be significant, when taking into account human capital and
technical infrastructure \cite{krauss2019statisticalArbitrage}. 