\section{Methodology}
Similarly to <Krauss et al. (2017)> and <cite arbitrage>, the methodology of this paper consists of
the following steps:

\begin{enumerate}
    \item The entire data set is split into a training, a validation and a trading set.
    \item The respective features (explanatory variables) and targets (dependent variables) are created
    \item Each model is trained on the training set
    \item Conduct out-of-sample predictions on the trading set for each model 
    \item Evaluate its accuracy and trading-performance on the trading set respectively
    \item Go to Step 2, and repeat the same steps for a different feature- and target-specification
  \end{enumerate}


\subsection{Training and Trading Set}
In our application to minute-binned data, the test set, i.e. trading set, contains all observations from 01.11.2019 to 31.12.2019.
The training set ranges from 01.01.2019 to 14.09.2019, and the remaining 15.09.2019 to 31.10.2019 
is reserved for the valdation set. We decided against the usual k-fold cross-validation approach 
in order to emphasize the importance of future observation for the model, since its performance only gets
evaluated on the future trading set.

\subsection{Feature and Target Generation}
Broadly following <cite>, we generate the feature space as follows:

\begin{description}
    \item[Input:] Let $ P^{c} = ( P^{c}_{t} )_{t \in T} $ denote the price process of coin-USD-pair $ c $, with $ c \in \{1, ... , n\} $. The price itself is the average between \textit{Open} and \textit{Close}.
    \item[Features:] From the data set we obtain the following features:
    \begin{description}
        \item[Returns:] Let $ R^{c}_{t, t - m} $ be the simple return for coin $ c $ over $ m $ periods defined as
        \begin{equation}
            R^{c}_{t, t - m} = \frac{ P^{c}_{t} }{ P^{c}_{t - m} } - 1 
        \end{equation} 
        \item[Volumes:] Let $ V^{c}_{t} $ be the traded volume for coin $ c $ in minute-bin $ t $ scaled by Quantile-Transformer fitted separately for each coin
    \end{description}
    \item[Target:] Let $ Y^{c}_{t + 1, t} $ be a binary response variable for each coin $c$. It assumes value 1 (class \textit{up}) if its future 120 min return $ R^{c}_{t + 120, t + 1} $ is greater than its cross-sectional median across all pairs $ ( R^{c}_{t + 120, t + 1} )^{n}_{c=1} $, else -1 (class \textit{down}).
\end{description}


We decided for the inclusion of volume such that the model has a measure for 
taking trading activity into account without breaking vital assumptions needed for testing the 1. Efficient Market Hypothesis (<cite>).
In addition, the volume got scaled for each coin in order to make the measure more comparable across coins,
since we are training a single universal model for each of the selected coins. 
Further, the Quantile-Transformation handles outliers (<cite>) and restricts the feature to an intervall ranging from 0 to 1.


\subsection{Model Training}

\subsubsection{Logistic Regression}
Lorem ipsum dolor sit amet, consetetur sadipscing elitr, sed diam nonumy eirmod tempor invidunt ut labore et dolore magna aliquyam erat, sed diam voluptua.

\subsubsection{Random Forest}
Lorem ipsum dolor sit amet, consetetur sadipscing elitr, sed diam nonumy eirmod tempor invidunt ut labore et dolore magna aliquyam erat, sed diam voluptua.

\subsubsection{Support Vector Machine}
Lorem ipsum dolor sit amet, consetetur sadipscing elitr, sed diam nonumy eirmod tempor invidunt ut labore et dolore magna aliquyam erat, sed diam voluptua.

\subsubsection{Artificial Neural Network}
Lorem ipsum dolor sit amet, consetetur sadipscing elitr, sed diam nonumy eirmod tempor invidunt ut labore et dolore magna aliquyam erat, sed diam voluptua.


\subsection{Trading}
Lorem ipsum dolor sit amet, consetetur sadipscing elitr, sed diam nonumy eirmod tempor invidunt ut labore et dolore magna aliquyam erat, sed diam voluptua.

