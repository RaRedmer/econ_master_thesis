\section{Introduction}
In the realm of finance literature, it is well known that financial time series
are notoriously difficult to predict, primarily driven by the high degree of noise \cite{fischer2017lstmMarketPrediction}.
Moreover, the generally accepted weak efficient market hypothesis provides a theoretical framework
which encompasses this phenomenon \cite{fama1970marketHypothesis}.
The weak form states that current asset prices reflect all the information of past prices 
and that no form of analysis can be effectively utilized to consistently support investors in making trading decisions,
thus eliminating arbitrage solely based on price data.
Therefore, also machine learning algorithms which were trained on price data 
should not be able to do so according to this hypothesis. 
In order to test this hypothesis, I trained multiple machine learning models on price data with different feature and target
specifications to predict price movements. I chose a classification instead of a regression problem, as the literature
suggests that it performs better for predicting financial market data
(\cite{leung2000classificationStockIndices}, \cite{enke2005classificationNN}).
Then, I used the estimated probabilities from the models in a backtest 
to simulate actual trading with a trading algorithm.
Model training and backtest are conducted on minute-binned OHLC-data of cryptocurrency coins from the 
Bitfinex exchange. The reason for choosing cryptocurrency assets is the fact that 
they have remained fairly unregulated by governmental institutions 
(\cite{dyhrberg2015bitcoinRegulations}, \cite{houben1994cryptoRegulation}).
Therefore, this asset class and its exchanges are more in line with the underlying assumption of the
efficient market hypothesis such as perfect markets, thus I expected that arbitrage opportunities only rarely occur. 
This expectation is further supported by low entry barriers and transaction costs \cite{bitfinex2012}.

This work extends the existing literature by

\begin{itemize}
    \item including volumes traded in the respective minute-bin,
    \item comparing the effects of different durations,
    \item explaining in detail how transaction costs got incorporated,
    \item analysing how changing the probability threshold affects returns and
    \item improving the trading algorithm by not making it automatically close an active position, regardless of probability signal.
\end{itemize}


\section{Literature Review}
The authors of \cite{shah2014bayesianRegression} predicted price changes with a Bayesian regression model of Bitcoin during a six month period in 2014. 
The model performed exceptionally well with a return of 89 percent and a Sharpe ratio of
4.10 during a period of 50 trading days.
However, no transaction costs were not incorporated and
they assumed perfect liquidity. 
\cite{takeuchi2013momentumTrading} used a momentum strategy on the U.S. CRSP
stock universe ranging from 1965 to 2009. This strategy was based on deep neural network
classifiers to estimate the probability for each stock to outperform the cross-sectional median
return of all stocks in the holding month $t + 1$.
\cite{dixon2015annMarketPrediction} achieved a relatively high classification accuracy of 73
percent with five-minute binned return data. 
However, they did not take microstructural market effects into account which is essential when dealing with
high-frequency data.
\cite{moritz2014partFutureReturns} trained a random forests on U.S. CRSP data ranging from 1968 to
2012 to incorporate it into a trading strategy. 
\cite{krauss2019statisticalArbitrage} trained a random
forest on the lagged returns of 40 cryptocurrency coins to predict whether a coin
outperforms the cross-sectional median of 40 coins over a forecast period of 120 min. 
They enter a long position with the top-3 coins
and a long position with the bottom-3 coins in terms of prediction probability. After 120 min, they reverse the
respective positions. During the out-of-sample period of their backtest, ranging from 18 June 2018
to 17 September 2018, and after more than 100,000 trades, they find statistically and economically
significant returns of 7.1 bps per day, after transaction costs of 15 bps per half-turn.
\cite{krauss2016arbitrageSandP} analysed the effectiveness of deep neural networks, 
gradientboosted trees, Random Forests, and a combination of these methods for identifying
statistical arbitrage. Each model got trained on lagged returns of all stocks in
the S\&P 500. Daily one-day-ahead
trading signals are generated based on the probability forecast of a stock to outperform
the market.